\documentclass[12pt]{beamer}
\usetheme{Copenhagen}
\usepackage[latin1]{inputenc}
\usepackage{amsmath}
\usepackage{amsfonts}
\usepackage{amssymb}
\usepackage{graphicx}
\author{Equipo 5}
\author{Equipo 5}
\title{Prediccion porcentaje de impurezas}

\begin{document}

\begin{frame}
\titlepage
\end{frame}



\begin{frame}{Problematica}
En una empresa cosmetiquera, se analiza el estado de las sustancias para detectar impurezas en su estado y estudiar que tan factible es el emplear estos compuestos.
\\

Se sospecha que en estos procesos quimicos hay una relacion lineal entre temperatura y el porcentaje de impureza.
\\

Nuestras variables a utilizar son temperatura en grados Celsius, que identificaremos con
la letra 'x' y el porcentaje de impureza que se representara con la letra 'y'.
\\

Se utilizara software como MatLab para la comprobacion de nuestros procedimientos y Excel para hacer los calculos pertinentes y la graficacion de nuestro metodo para una
mejor visualizacion.

\end{frame}

\begin{frame}{Metodologia}

Nuestras variables a utilizar son temperatura en
grados Celsius, que identificaremos con la letra
'x' y el porcentaje de impureza que se
representara con la letra 'y'.
Utilizaremos el software MatLab para la
comprobacion de nuestros procedimientos y
Excel para hacer los calculos pertinentes y la
graficacion de nuestro metodo para una mejor
visualizacion.

\end{frame}


\begin{frame}{Metodo de regresion lineal}

Es un metodo estadistico que estudia la relacion lineal
existente entre dos variables. Consiste en generar un modelo o
una ecuacion, que basado en la relacion de ambas variables
permita predecir el valor de una a partir de la otra. El modelo es
cambiante de acuerdo a la variable que se considere
dependiente de la otra. En un nivel experimental es muy comun
que una variable sea controlable y la otra sea medida.

\end{frame}

\begin{frame}{Resultados excel}
\includegraphics[scale=0.3]{pres1.PNG} 

\end{frame}

\begin{frame}{Resultados Matlab}
\includegraphics[scale=0.3]{pres2.jpeg} 

\end{frame}

\begin{frame}{Conclusion}

El coeficiente de determinacion indica que la
variabilidad de los datos observados es poca y
siguen una tendencia se ajustan muy bien a la
linea recta, y esto indica tambien que es una
buena relacion lineal.
\\

Con un NS se rechaza la H0, es decir la
pendiente poblacional de la temperatura
respecto al porcentaje de impureza si es
diferente a 0, hay relacion lineal entre estas dos
variables

\end{frame}

\end{document}